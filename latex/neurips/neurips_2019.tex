\documentclass{article}

% if you need to pass options to natbib, use, e.g.:
%     \PassOptionsToPackage{numbers, compress}{natbib}
% before loading neurips_2019

% ready for submission
% \usepackage{neurips_2019}

% to compile a preprint version, e.g., for submission to arXiv, add add the
% [preprint] option:
%     \usepackage[preprint]{neurips_2019}

% to compile a camera-ready version, add the [final] option, e.g.:
\usepackage{neurips_2019}

% to avoid loading the natbib package, add option nonatbib:
%     \usepackage[nonatbib]{neurips_2019}

\usepackage[utf8]{inputenc} % allow utf-8 input
\usepackage[T1]{fontenc}    % use 8-bit T1 fonts
\usepackage{hyperref}       % hyperlinks
\usepackage{url}            % simple URL typesetting
\usepackage{booktabs}       % professional-quality tables
\usepackage{amsfonts}       % blackboard math symbols
\usepackage{nicefrac}       % compact symbols for 1/2, etc.
\usepackage{microtype}      % microtypography
\usepackage{graphviz}
\usepackage{auto-pst-pdf}
\usepackage{listings}
\usepackage{float}
\usepackage{xcolor}
\usepackage{framed}
% Code listings
%\usepackage[scaled]{beramono}
\usepackage{fontspec}
\setmonofont{JetBrains Mono}
\usepackage{ocr}
\usepackage[skins,breakable,listings]{tcolorbox}

\usepackage{accsupp}
\newcommand{\noncopyable}[1]{%
    \BeginAccSupp{method=escape,ActualText={}}%
    #1%
    \EndAccSupp{}%
}

\lstdefinelanguage{kotlin}{
    comment=[l]{//},
    commentstyle={\color{gray}\ttfamily},
    emph={delegate, filter, firstOrNull, forEach, it, lazy, mapNotNull, println, repeat, assert, with, head, tail, len, return@},
    numberstyle=\noncopyable,
    emphstyle={\color{OrangeRed}},
    identifierstyle=\color{black},
    keywords={abstract, actual, as, as?, break, by, class, companion, continue, data, do, dynamic, else, enum, expect, false, final, for, fun, get, if, import, in, infix, interface, internal, is, null, object, open, operator, override, package, private, public, return, sealed, set, super, suspend, this, throw, true, try, typealias, val, var, vararg, when, where, while, tailrec, reified},
    keywordstyle={\color{NavyBlue}\bfseries},
    morecomment=[s]{/*}{*/},
    morestring=[b]",
    morestring=[s]{"""*}{*"""},
    ndkeywords={@Deprecated, @JvmField, @JvmName, @JvmOverloads, @JvmStatic, @JvmSynthetic, Array, Byte, Double, Float, Boolean, Int, Integer, Iterable, Long, Runnable, Short, String},
    ndkeywordstyle={\color{BurntOrange}\bfseries},
    sensitive=true,
    stringstyle={\color{ForestGreen}\ttfamily},
    literate={`}{{\char0}}1
}

\usepackage{lipsum}
\tcbuselibrary{listings,breakable}

\newtcblisting{javalisting}[1][]{%
    breakable=false,
    listing only,
    boxsep=-1.5pt,
    top=-1pt,
    bottom=-0.5pt,
    listing options={
        language=kotlin,
        basicstyle=\ttfamily\tiny,
        numberstyle=\footnotesize\noncopyable,
        tabsize=2,
        numbers=right,
        breaklines=true,
        inputencoding=utf8,
        escapeinside={(*@}{@*)},
        #1
    },
}

\tcbset{
    enhanced jigsaw,
    breakable,
    listing only,
    boxsep=-1pt,
    top=-1pt,
    bottom=-0.5pt,
    right=-0.5pt,
    overlay first={
        \node[black!50] (S) at (frame.south) {\Large\ding{34}};
        \draw[dashed,black!50] (frame.south west) -- (S) -- (frame.south east);
    },
    overlay middle={
        \node[black!50] (S) at (frame.south) {\Large\ding{34}};
        \draw[dashed,black!50] (frame.south west) -- (S) -- (frame.south east);
        \node[black!50] (S) at (frame.north) {\Large\ding{34}};
        \draw[dashed,black!50] (frame.north west) -- (S) -- (frame.north east);
    },
    overlay last={
        \node[black!50] (S) at (frame.north) {\Large\ding{34}};
        \draw[dashed,black!50] (frame.north west) -- (S) -- (frame.north east);
    },
    before={\par\vspace{10pt}},
    after={\par\vspace{\parskip}\noindent}
}

\newcommand*{\inlineimg}[1]{%
    \raisebox{-.3\baselineskip}{%
        \includegraphics[
            height=\baselineskip,
            width=\baselineskip,
            keepaspectratio,
        ]{#1}%
    }%
}

\definecolor{slightgray}{rgb}{0.90, 0.90, 0.90}

\usepackage{soul}
\makeatletter
\def\SOUL@hlpreamble{%
    \setul{}{3.0ex}%
    \let\SOUL@stcolor\SOUL@hlcolor%
    \SOUL@stpreamble%
}
\makeatother

\newcommand{\inline}[1]{%
    \begingroup%
    \sethlcolor{slightgray}%
    \hl{\ttfamily\footnotesize #1}%
    \endgroup
}

\newcommand{\tinline}[1]{%
    \begingroup%
    \sethlcolor{slightgray}%
    \hl{\ttfamily\tiny #1}%
    \endgroup
}

\title{A Graph-Based Intermediate Representation\\for Java Code and Developer Documentation}

\author{Breandan Considine\\
breandan.considine@mail.mcgill.ca\\
McGill University}

\begin{document}

\maketitle

\begin{abstract}
Semantic information plays a key role in the code search and synthesis settings. In this work, we propose a graph-based representation for source code and natural language which incorporates semantic and relational features from both domains. We apply this graph to parsing a corpus of Java code and developer docs, and demonstrate the effectiveness of a common graph-based representation on three downstream tasks: code search, document recommendation and link prediction.
\end{abstract}

\section{Background and motivation}

In addition to its syntactic structure, source code contains a rich denotational and operational semantics~\citep{henkel2018code}. To effectively reason about code in semantically similar but syntactically diverse settings requires models which incorporate features from the call graph~\citep{gu2016deep, liu2019neural} and surrounding typing context~\citep{allamanis2017learning}. Many semantic features, such as data and control flow~\citep{si2018learning} can be represented as a directed acyclic graph (DAG), which admits linear-time solutions to a number of graph problems, including topological sorting, single-source shortest path and reachability queries.

Natural language also contains semantic information which can be represented using graphs. The NLP community has a rich set of graph based representations, including \citet{reddy2016transforming}'s and other typed attribute grammars which can be used to reason about syntactic and semantic relations. The pointer network architecture~\citep{vinyals2015pointer, vinyals2015order} can be used to construct permutation-invariant semantic relations between natural language entities, and have important applications in dependency parsing~\citep{ma2018stack}, named-entity recognition~\citep{lample2016neural}, and other language modeling tasks where sequence-based representations fall short. \citet{li2017code} extend pointer networks with a copy-mechanism to handle out-of-vocabulary code tokens.

Content recommendation for doc-to-doc (D2D) and code-to-code (C2C) settings can be solved with existing link prediction~\citep{zhang2018link} and code embedding~\citep{gu2018deep} techniques, but cross-domain transfer remains challenging. \citet{robillard2015recommending, robillard2017demand} first explore the task of predicting reference API docs from source code using manual annotation. Prior work also studies the association between comments and code entities~\citep{panthaplackel2020associating} using machine learning, but only within source code.

Maintainers of widely-used software projects often publish web-based developer docs, typically stored in markup languages like HTML or Markdown. These files contain a collection of natural language sentences, markup, and hyperlinks to other documents. Both the link graph and the document tree contain important semantic information: the markup describes the text in relation to the other entities in the document hierarchy~\citep{yang2016hierarchical}, while the link graph describes the relationship between the parent document and related documents or source code entities. Documents occasionally contain hyperlinks to source code, but source code rarely contains links to developer documents.

Some programming languages allow users to specify which type of values will inhabit a given variable at runtime. Types allow the compiler to reason about certain properties like nullity~\citep{ekman2007pluggable} and shape~\citep{considine2019kotlingrad}. While types many not appear explicitly in source code, they can often be inferred from the surrounding context using a dataflow graph (DFG). The Java language recently introduced local variable type inference~\citet{liddell2019analyzing}, which allows variable types to be omitted, and later inferred by the compiler.

\section{Proposed approach}

Given a single token in either source code or developer documentation and its surrounding context, what are the most relevant source code or documentation entities related to the token? We would like to infer which entities are connected to a given token, based on the surrounding semantic context. Learning relationships between these two domains requires parsing both natural language and source code. Following \citet{si2018learning, gu2018deep, liu2019neural}, we use a node embedding on the dataflow graph and type environment, and following \citet{yang2016hierarchical, zhang2018link}, use the markup hierarchy and link graph to construct an embedding for code-like tokens used within documentation.

To compensate for the sparsity of hyperlinks between code and documentation, we need to construct bridge links between the documentation graph and source code entities. One heuristic which developers often use to discover relevant documents is plaintext search on a salient lexical string. Co-occurrence of an uncommon token is unlikely to be coincidence, and indicates that two entities are likely related, even though they may not share an explicit grammatical link. If we can recover this relationship using semantic and type-related information without observing the lexical token itself, this is a further indication our representation is providing useful information.

\section{Data availability}

Java, one of the most prolific programming languages on GitHub, is a statically typed language with a high volume of API documentation. With has a variety of tools source code~\citep{parr2013definitive, hosseini2013javaparser, kovalenko2019pathminer} and natural language~\citep{manning2014stanford, grella2018non} parsing, it is a convenient language for both analysis and implementation. Our dataset consists of Java repositories on GitHub, and their accompanying developer documents. All projects in our dataset have a collection of source code files and multiple related repositories on GitHub.

We construct two datasets consisting of naturally-occurring links between developer docs and source code, and a surrogate set of links constructed by matching lexical tokens available in both domains. Our target is recovery of ground truth links in the test set and surrogate links in the lexical matching graph. By adding weighted edges between source code and docs, we evaluate our approach by predicting synthetic links between tokens contained in code fragments and markup entities which refer to the selected token. In addition, we evaluate our approach on both D2D and C2C link retrieval, as well as precision and recall on the surrogate link relations.

Our data consists of two complementary datasets: abstract syntax trees collected from Java source code and developer documentation. We use the astminer~\citep{kovalenko2019pathminer} library to parse Java code, jsoup~\citep{hedley2009jsoup} to parse HTML and Stanford's CoreNLP~\citep{manning2014stanford} library to parse dependency graphs from developer docs. Consider the following AST, parsed from the \href{https://www.eclipse.org/collections/}{Eclipse Collections} Java project:

\vspace{20pt}

\begin{figure}[H]
    \centering
    \digraph[scale=0.5]{ast} {
    0 -> {1 2 3};
    1 -> {4 5 6};
    4 -> {};
    5 -> {7 8 9 10 11 12 13};
    7 -> {};
    8 -> {};
    9 -> {};
    10 -> {};
    11 -> {};
    12 -> {};
    13 -> {};
    6 -> {};
    2 -> {14 15};
    14 -> {};
    15 -> {16 17 18};
    16 -> {};
    17 -> {};
    18 -> {19 20};
    19 -> {};
    20 -> {};
    3 -> {};
}
%    \caption{This AST was parsed from the following code snippet:}
    \label{fig:ast}
\end{figure}

\vspace{-30pt}The AST depicted above was generated by parsing the following code snippet:

\begin{javalisting}
public void lastKey_throws() {
    new ImmutableTreeMap<>(new TreeSortedMap<>()).lastKey();
}
\end{javalisting}

Now consider the following dependency graph, taken from a Javadoc in the same project:

\begin{figure}[H]
    \centering
    \digraph[scale=0.17]{eng} {
    N_1 [label="The/DT-1"];
    N_2 [label="advantages/NNS-2"];
    N_3 [label="to/TO-3"];
    N_4 [label="using/VBG-4"];
    N_5 [label="this/DT-5"];
    N_6 [label="container/NN-6"];
    N_7 [label="over/IN-7"];
    N_8 [label="a/DT-8"];
    N_9 [label="Map/NN-9"];
    N_10 [label="</JJR-10"];
    N_11 [label="K/NN-11"];
    N_12 [label=",/,-12"];
    N_13 [label="Collection/NNP-13"];
    N_14 [label="<V>/NNP-14"];
    N_15 [label=">/JJR-15"];
    N_16 [label="is/VBZ-16"];
    N_17 [label="that/IN-17"];
    N_18 [label="all/DT-18"];
    N_19 [label="of/IN-19"];
    N_20 [label="the/DT-20"];
    N_21 [label="handling/NN-21"];
    N_22 [label="of/IN-22"];
    N_23 [label="the/DT-23"];
    N_24 [label="value/NN-24"];
    N_25 [label="collection/NN-25"];
    N_26 [label="can/MD-26"];
    N_27 [label="be/VB-27"];
    N_28 [label="done/VBN-28"];
    N_29 [label="automatically/RB-29"];
    N_30 [label="./.-30"];
    N_31 [label="It/PRP-31"];
    N_32 [label="also/RB-32"];
    N_33 [label="allows/VBZ-33"];
    N_34 [label="implementations/NNS-34"];
    N_35 [label="to/TO-35"];
    N_36 [label="further/RBR-36"];
    N_37 [label="specialize/VB-37"];
    N_38 [label="in/IN-38"];
    N_39 [label="how/WRB-39"];
    N_40 [label="duplicate/VB-40"];
    N_41 [label="values/NNS-41"];
    N_42 [label="will/MD-42"];
    N_43 [label="be/VB-43"];
    N_44 [label="handled/VBN-44"];
    N_45 [label="./.-45"];
    N_46 [label="Value/NNP-46"];
    N_47 [label="collections/NNS-47"];
    N_48 [label="with/IN-48"];
    N_49 [label="list/NN-49"];
    N_50 [label="semantics/NNS-50"];
    N_51 [label="would/MD-51"];
    N_52 [label="allow/VB-52"];
    N_53 [label="duplicate/VB-53"];
    N_54 [label="values/NNS-54"];
    N_55 [label="for/IN-55"];
    N_56 [label="a/DT-56"];
    N_57 [label="key/NN-57"];
    N_58 [label=",/,-58"];
    N_59 [label="while/IN-59"];
    N_60 [label="those/DT-60"];
    N_61 [label="implementing/VBG-61"];
    N_62 [label="set/NN-62"];
    N_63 [label="semantics/NNS-63"];
    N_64 [label="would/MD-64"];
    N_65 [label="not/RB-65"];
    N_66 [label="./.-66"];
    N_2 -> N_1 [label="det"];
    N_2 -> N_66 [label="punct"];
    N_2 -> N_4 [label="acl:to"];
    N_2 -> N_52 [label="parataxis"];
    N_2 -> N_45 [label="punct"];
    N_4 -> N_3 [label="mark"];
    N_4 -> N_6 [label="dobj"];
    N_4 -> N_9 [label="nmod:over"];
    N_6 -> N_5 [label="det"];
    N_9 -> N_7 [label="case"];
    N_9 -> N_8 [label="det"];
    N_9 -> N_11 [label="nmod:tmod"];
    N_11 -> N_10 [label="amod"];
    N_11 -> N_12 [label="punct"];
    N_11 -> N_14 [label="appos"];
    N_11 -> N_15 [label="amod"];
    N_14 -> N_13 [label="compound"];
    N_15 -> N_16 [label="dep"];
    N_16 -> N_28 [label="ccomp"];
    N_18 -> N_21 [label="nmod:of"];
    N_21 -> N_19 [label="case"];
    N_21 -> N_20 [label="det"];
    N_21 -> N_25 [label="nmod:of"];
    N_25 -> N_22 [label="case"];
    N_25 -> N_23 [label="det"];
    N_25 -> N_24 [label="compound"];
    N_28 -> N_17 [label="mark"];
    N_28 -> N_33 [label="parataxis"];
    N_28 -> N_18 [label="nsubjpass"];
    N_28 -> N_26 [label="aux"];
    N_28 -> N_27 [label="auxpass"];
    N_28 -> N_29 [label="advmod"];
    N_28 -> N_30 [label="punct"];
    N_33 -> N_32 [label="advmod"];
    N_33 -> N_34 [label="dobj"];
    N_33 -> N_37 [label="xcomp"];
    N_33 -> N_31 [label="nsubj"];
    N_37 -> N_34 [label="nsubj:xsubj"];
    N_37 -> N_35 [label="mark"];
    N_37 -> N_36 [label="advmod"];
    N_37 -> N_44 [label="advcl:in"];
    N_40 -> N_39 [label="advmod"];
    N_44 -> N_38 [label="mark"];
    N_44 -> N_40 [label="dep"];
    N_44 -> N_41 [label="nsubjpass"];
    N_44 -> N_42 [label="aux"];
    N_44 -> N_43 [label="auxpass"];
    N_47 -> N_50 [label="nmod:with"];
    N_47 -> N_46 [label="compound"];
    N_50 -> N_48 [label="case"];
    N_50 -> N_49 [label="compound"];
    N_52 -> N_65 [label="advcl:while"];
    N_52 -> N_51 [label="aux"];
    N_52 -> N_53 [label="xcomp"];
    N_52 -> N_58 [label="punct"];
    N_52 -> N_47 [label="nsubj"];
    N_53 -> N_54 [label="dobj"];
    N_53 -> N_57 [label="nmod:for"];
    N_57 -> N_55 [label="case"];
    N_57 -> N_56 [label="det"];
    N_60 -> N_61 [label="acl"];
    N_61 -> N_63 [label="dobj"];
    N_63 -> N_62 [label="compound"];
    N_65 -> N_64 [label="aux"];
    N_65 -> N_59 [label="mark"];
    N_65 -> N_60 [label="nsubj"];
}
    \caption{This graph was parsed from the following comment:
        ``The advantages to using this container over a \footnotesize\texttt{Map<K, Collection<V>{}>} is
        that all of the handling of the value collection can be done automatically.
        It also allows implementations to further specialize in how duplicate values
        will be handled. Value collections with list semantics would allow
        duplicate values for a key, while those implementing set semantics would not.
        The value collections can never be empty.''
    }
    \label{fig:eng}
\end{figure}

Our goal is to connect these two graphs using a common representation for source code and natural language. Absent any explicit \texttt{@link} or \texttt{@see} annotations, in order to relate these two graphs, we must somehow infer the shared semantic entities, which we can do for a subset using a simple lexical matching procedure.

\section{Experiment}

We then train four link prediction models on the following datasets:

\begin{itemize}
    \item Code graph (CG)
    \item Documentation graph (DG)
    \item Both code and documentation graphs separately (CDG)
    \item Single graph containing code and documentation entities related by lexical matching (LEX)
\end{itemize}

For each model, we evaluate our trained model on link prediction in the same setting and every other setting.

\clearpage
\newpage

\bibliography{neurips_2019}
\bibliographystyle{plainnat}
\end{document}
